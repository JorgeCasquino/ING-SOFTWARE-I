\documentclass{article}
\usepackage[utf8]{inputenc}
\usepackage{hyperref}
\usepackage{natbib}

\title{Evaluación de Software mediante el Índice de Seguridad del Código}
\author{Casquino Ticona Jorge Romel}
\date{Junio 2024}

\begin{document}

\maketitle

\begin{abstract}
El Índice de Seguridad del Código es una métrica crucial para evaluar la robustez y seguridad de las aplicaciones de software. Este documento explora la definición, aplicaciones y estudios recientes relacionados con esta métrica, destacando su importancia en la protección contra vulnerabilidades y amenazas cibernéticas.
\end{abstract}

\section{Introducción}
El Índice de Seguridad del Código evalúa la seguridad de una aplicación de software basándose en la identificación y mitigación de vulnerabilidades. Esta métrica es fundamental para asegurar que las aplicaciones sean resistentes a ataques y cumplan con los estándares de seguridad \citep{xu2023}.

\section{Definición y Métricas Relacionadas}
El Índice de Seguridad del Código incluye diversas medidas que evalúan diferentes aspectos de la seguridad del software:
\begin{itemize}
    \item \textbf{Vulnerabilidades Detectadas}: Número de vulnerabilidades encontradas en el código.
    \item \textbf{Parches Aplicados}: Eficiencia y rapidez en la aplicación de parches de seguridad.
    \item \textbf{Evaluaciones de Seguridad}: Resultados de pruebas de penetración y auditorías de seguridad \citep{inderscience2023}.
\end{itemize}

\section{Aplicaciones y Limitaciones}
Estas métricas se utilizan para evaluar la seguridad de los sistemas de software y la eficacia de las prácticas de desarrollo seguro. Sin embargo, es importante utilizarlas en conjunto con otras evaluaciones de seguridad para obtener una visión completa \citep{nature2023}.

\section{Estudios Recientes}
Un estudio reciente en el \textit{Journal of Cybersecurity and Privacy} analiza diversos enfoques para la gestión de riesgos cibernéticos y la implementación de marcos de inteligencia de amenazas para proteger infraestructuras críticas \citep{mdpi2023}. Otro estudio en la \textit{International Journal of Information and Computer Security} discute los desafíos y soluciones en la ingeniería de requisitos de software ágil, destacando la importancia de alinear prácticas de seguridad con el desarrollo ágil \citep{inderscience2023}.

\section{Conclusión}
El Índice de Seguridad del Código proporciona una evaluación integral de la seguridad del software. Complementar esta métrica con otras evaluaciones de seguridad es esencial para proteger las aplicaciones contra amenazas cibernéticas.

\bibliographystyle{plainnat}
\bibliography{references}

\end{document}
